\section{Quoting}

Generally, quoting is done by remaining faithful to the original text in terms of words, spelling and punctuation. In case there is a mistake, the correct version is written in square brackets in the quoted text.

Short quotations (not longer than 40 words) must be given in quotation marks. Following the text quoted, the reference must be written and a full-stop must be placed afterwards.  

Quotations longer than 40 words must not be shown in quotation  marks. Instead, they must be indented 1 tab space (1.27 cm) from the left side of the page. The font size for long quotations indented from the left must be 2 pt smaller than the font size used in main text body. However, it is not advised to quote very long texts and to quote very frequently. Unlike short quotations, references of long quotations must be placed after the full stop. (i.e., .(p.196))

Example for a quotation at the beginning of a sentence;

According to Jones (1998), "Students often had difficulty using APA style,  especially when it was their first time" (p. 199).

Example for a quotation in the middle of a sentence;

Interpreting these results, Robbins et al. (2003) suggested that the “therapists in dropout cases may have inadvertently validated parental negativity about the adolescent without adequately responding to the adolescent’s needs or concerns” (p. 541) contributing to an overall climate of negativity.

Example for a quotation at the end of a sentence;

Confusing this issue is the overlapping nature of roles in palliative care, whereby “medical needs are met by those in the medical disciplines; nonmedical needs may be addressed by anyone on the team” (Csikai \& Chaitin, 2006, p. 112). 

Detailed information on quoting could be found on websites of Graduate Schools and associated links.